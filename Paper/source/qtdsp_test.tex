\section{Testing}

The demodulation algorithm was tested by coding the algorithm in C (as a console
application) and instrumenting it to display variables,
save arrays to files, and benchmark the various stages.
KissFFT is used as the 1K complex FFT.
On a fast Core-i7 PC, the algorithm correlates a frame in about 100 usec.
Allowing 100 usec per 32 points amounts to 320K SPS of input which, when $R=-0.5$,
is 80K SPS of output.

\subsection{EEG data}

An ideal test for the algorithm is EEG data taken from public data sets.
EDF is the preferred format since it's documented well enough to be useful
and fairly common. 
An EDF parsing library (in C) is at https://gitlab.com/Teuniz/EDFlib.

The demodulation algorithm looks at the signal bandwidth between about
$F_S/5$ and $F_S/2$.
Many data sets have a region of interest far below the sample rate.
Decimation is typically used to lower the sample rate to bring it to the
desired region of interest. 
Decimation by a factor of M usually involves low-pass filtering the signal
and taking every Mth data point as output.
A more sloppy method is to simply sum each set of M input points and ignore
the aliasing effects since those will be smeared across the noise floor.

A graphic display can have a waterfall representation of the waveform along
with cue markers. Once processed, the graphic image can be panned or scrolled
as appropriate within the app or output as a video file for playback.


