\section{Testing}

The demodulation algorithm was tested by coding the algorithm in C (as a console
application) and instrumenting it to display variables,
save arrays to files, and benchmark the various stages.
KissFFT is used as the 1K complex FFT.
On a fast Core-i7 PC, the algorithm correlates a frame in about 100 usec.
Allowing 100 usec per 32 points amounts to 320K SPS of input which, when $R=-0.5$,
is 80K SPS of output.

\subsection{EEG data}

An ideal test for the algorithm is EEG data taken from public data sets.
https://www.physionet.org/pn6/chbmit/ has EEG datasets with epileptic seizures.
Two channels of data (1 and 13) were extracted from file chb03_01.edf (40MB).
These correspond to the left and right sides of the forehead.
3600 seconds of 256 Hz data provides 921600 samples per channel.
A seizure starts at 362 seconds and ends at 414 seconds.

The demodulation algorithm looks at the signal bandwidth between about
$F_S/5$ and $F_S/2$.
Many data sets have a region of interest far below the sample rate.

Decimation is typically used to lower the sample rate to bring it to the
desired region of interest. 
Decimation by a factor of $m$ involves low-pass filtering the signal
and taking every $m$th data point as output.
The spectrogram of the test data shows little information above 30 Hz.
The 256 Hz data is downsampled to 64 Hz ($m=4$).

The demodulation app processes up to a minute or so of data starting at a 
beginning time and saves it as a BMP (using colored amplitude) with time on the
horizontal axis and R on the vertical.
It's feasible to implement just a console application,
avoiding GUI work altogether.
The BMP is the scientific output, viewable in many media as well as on the web.
For an R range of 0.25 to 0.75, there are between 3 and 8 input points per 
output point. 

